%%%%%%%%%%%%%%%%%%%%%%%%%%%%%%%%%%%%%%%%%
% Long Sectioned Curriculum Vitae
% LaTeX Template
% Version 1.1 (9/12/12)
%
% This template has been downloaded from:
% http://www.latextemplates.com
%
% Original author:
% Rensselaer Polytechnic Institute (http://www.rpi.edu/dept/arc/training/latex/resumes/)
%
% Important note:
% This template requires the res.cls file to be in the same directory as the
% .tex file. The res.cls file provides the resume style used for structuring the
% document.
%
%%%%%%%%%%%%%%%%%%%%%%%%%%%%%%%%%%%%%%%%%

%----------------------------------------------------------------------------------------
%	PACKAGES AND OTHER DOCUMENT CONFIGURATIONS
%----------------------------------------------------------------------------------------

\documentclass{res} % Use the res.cls style, the font size can be changed to 11pt or 12pt here

\setlength{\resumewidth}{7in}
\usepackage[left=0.75in,
           top=0.75in,
           bottom=0.75in,
            ]{geometry}
\usepackage{multicol}
\usepackage{booktabs}
\usepackage{setspace}
%\usepackage{helvet} % Default font is the helvetica postscript font
\usepackage{newcent} % To change the default font to the new century schoolbook postscript font uncomment this line and comment the one above
\usepackage[usenames,dvipsnames,svgnames,table]{xcolor}
\definecolor{light-gray}{gray}{0.8}
\newcommand{\sectionRule}{{\vspace{-7pt} \color{light-gray} \hrulefill}}
\newsectionwidth{0pt} % Stops section indenting

\newcommand{\helv}[1]{{\fontfamily{phv}\selectfont #1}}


\makeatletter
\renewcommand{\paragraph}{%
  \@startsection{paragraph}{4}%
  {\z@}{0.0ex \@plus 1ex \@minus .2ex}{-1em}%
  {\normalfont\normalsize\bfseries}%
}
\makeatother
\usepackage{enumitem}

\begin{document}

%----------------------------------------------------------------------------------------
%	YOUR NAME AND ADDRESS(ES) SECTION
%----------------------------------------------------------------------------------------

\begin{multicols}{2}
	{\Huge {Edwin Lee}}
	\vfill
	\columnbreak
	\begin{flushright}
		{720-402-2473}\\	
		{25095 Emerald Way}\\
		{Cashion, OK  73016}\\
		{edwin.lee@nrel.gov}
	\end{flushright}
\end{multicols}
\vspace{-0.45in}
\hrulefill
\vspace{-0.2in}
\begin{resume}

%----------------------------------------------------------------------------------------
%	OBJECTIVE SECTION
%----------------------------------------------------------------------------------------
\section{\centerline{\helv{OBJECTIVE}}}
\sectionRule
\vspace{-10pt} % Gap between title and text

To leverage my skills as a proficient programmer, mechanical engineer, and numerical analyst in the field of advanced building and system simulation. 

%----------------------------------------------------------------------------------------
%	EDUCATION SECTION
%----------------------------------------------------------------------------------------
\section{\centerline{\helv{EDUCATION}}} 
\sectionRule
\vspace{-8pt} % Gap between title and text

{\sl Doctor of Philosophy},
Mechanical Engineering \\
Oklahoma State University, Stillwater, OK \dotfill May 2013 \\
THESIS - A Generalized Pipe Heat Transfer Model for Whole Building Simulation Applications  \\
GPA 4.00

%----------------------------------------------------------------------------------------
%	PROFESSIONAL EXPERIENCE SECTION
%----------------------------------------------------------------------------------------
\section{\centerline{\helv{ENGINEERING EXPERIENCE}}} 
\sectionRule
\vspace{-8pt} % Gap between title and text

{\sl Research Engineer} \dotfill May 2013 -- Present \\
National Renewable Energy Laboratory, Golden, CO
\begin{itemize} \itemsep -2pt
	\item Contributed to the Technology Performance Exchange (TPEx) via Data Entry Form development, dataset processing, and development of the logic and scripts to convert TPEx datasets into components on the Building Component Library
	\item Began leading technical development of EnergyPlus, overseeing the technical changes accompanying the translation from FORTRAN to C++, and StarTeam to GitHub
\end{itemize}

{\sl Graduate Assistant} \dotfill January 2006 -- May 2013 \\
Oklahoma State University, Stillwater, OK
\begin{itemize} \itemsep -2pt
	\item A complete re-write of the EnergyPlus central plant simulation, including solution algorithms, pump model re-work, and updating component model design  
	\item Developed a generalized horizontal ground heat exchanger model that includes interaction with a basement zone, specifically for use with foundation heat exchangers
	\item Performed experimental measurement and modeling of transport delay phenomena in piping systems
	\item Worked closely with the Center for the Built Environment at University of California, Berkeley, providing simulation support for Underfloor Air Distribution System research with EnergyPlus
\end{itemize}

%----------------------------------------------------------------------------------------
%	PUBLICATIONS SECTION
%----------------------------------------------------------------------------------------
\section{\centerline{\helv{PUBLICATIONS}}}
\sectionRule
\vspace{2pt} % Gap between title and text

{
\leftskip 0.1in
\parindent -0.1in
\parskip 0.075in
\begin{spacing}{1.1}

Raftery, P., E. Lee, T. Webster, T. Hoyt and F. Bauman.  2014.  \textit{Effects of furniture and contents on peak cooling load}.  Energy and Buildings: 85:445-457.

Studer, D., K. Fleming, E. Lee and W. Livingood.  2014.  \textit{Enabling Detailed Energy Analyses via the Technology Performance Exchange}.  Proceedings of the ACEEE Summer Study, Pacific Grove, CA, USA.

Lee, E., D. Fisher and J. Spitler. 2013. \textit{Efficient Horizontal Ground Heat Exchanger Simulation with Zone Heat Balance Integration}. HVAC\&R Research: 19(3):307-323.

Lee, E. and D. Studer. 2013. \textit{TIP 287: Reducing Technology Evaluation Costs Through a Technology Performance Exchange. Deliverable 2.5: Draft Data Entry Forms}. NREL Report No. TP-5500-60219.

Xiong, Z., E. Lee and D. Fisher. 2013. \textit{Development of a Horizontal Slinky Ground Heat Exchanger Model}. ASHRAE Transactions: 119(2).

Chandrasekharan, R., E. Lee, D. Fisher and P. Deokar. 2013. \textit{An Enhanced Simulation Model for Building Envelopes with Phase Change Materials}. ASHRAE Transactions: 119(2).

Cullin, J., Spitler, J. and E. Lee. 2013. \textit{Preliminary Investigation of the Effect of Horizontal Piping on the Performance of a Vertical Ground Heat Exchanger System}. ASHRAE Transactions: 119(2):302-311.
\end{spacing}
}

\end{resume} 
\end{document}